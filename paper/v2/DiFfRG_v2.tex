%\documentclass[aps,twocolumn,prd,superscriptaddress]{revtex4-2}
\documentclass[preprint,10pt,sort&compress]{elsarticle}
\usepackage[a4paper, total={6.5in, 8.6in}]{geometry}

%%%%%%%%%%%%% Packages %%%%%%%%%%%%%

% general
\usepackage[utf8]{inputenc}

\usepackage{amsmath}
\usepackage{bbm}
\usepackage{mathtools}
\usepackage{amsfonts}
\usepackage{mathrsfs}
\usepackage{slashed}
\usepackage{tensor}
\usepackage{bbold}
\usepackage{MnSymbol}
\usepackage{textcomp}
\usepackage[export]{adjustbox}
\usepackage[]{mdframed}


% graphics and colors
\usepackage{graphicx}
\usepackage{float}
\usepackage{xcolor}
\usepackage{array}
\usepackage[abs]{overpic}

% floats+
\usepackage{placeins}
\usepackage{makecell}
\usepackage{subcaption}

% units and refs
\usepackage{xspace}
\usepackage{siunitx}
\usepackage{xfrac}
\usepackage{hyperref}
\usepackage[nameinlink]{cleveref}
\usepackage{appendix}


% code listings
\usepackage{listings}
\definecolor{backcolor}{rgb}{0.99,0.98,0.98}
\definecolor{string-color}{rgb}{0.3333, 0.5254, 0.345}
\definecolor{darkgrey}{rgb}{0.0627, 0.07, 0.082}
\definecolor{darkred}{rgb}{0.3, 0.05, 0.05}
\definecolor{codeblue}{rgb}{0.2,0.35,0.75}
\definecolor{codepurple}{rgb}{0.38,0.1,0.52}
\definecolor{codegray}{rgb}{0.5,0.5,0.5}
\definecolor{codegreen}{rgb}{0.05,0.3,0.05}
\definecolor{codered}{rgb}{0.6,0.2,0.1}
\definecolor{backgroundColour}{rgb}{0.99,0.99,0.98}
\lstdefinestyle{myStyle}{
    language = C++,
    basicstyle = {\ttfamily \small \color{darkgrey}},
    backgroundcolor = {\color{backcolor}},
	commentstyle=\color{codegreen},
    stringstyle = {\color{string-color}},
    keywordstyle = {\color{codeblue}},
    keywordstyle = [2]{\color{codepurple}},
    keywordstyle = [3]{\color{codered}},
    keywordstyle = [4]{\color{codegray}},
    keywordstyle = [5]{\color{codegreen}},
    otherkeywords = {<, >, :, ::, DiFfRG, constexpr, uint, size_t, &, get, vector, array, Tensor, Scalar, FunctionND},
    morekeywords = [2]{AbstractModel, FEFunctionDescriptor, VariableDescriptor, ExtractorDescriptor, ComponentDescriptor
	TimeStepperSUNDIALS_IDA, UMFPack, Point, real, AD, NoJacobians, FE_AD,LLFFlux,FlowBoundaries
	},
    morekeywords = [3]{DiFfRG, CG, DG, dDG, LDG, def, Variables, dealii, std, autodiff},
    morekeywords = [4]{<, >, :, ::, ;, &},
    morekeywords = [5]{},
	breakatwhitespace=false,
	breaklines=true,
	captionpos=b,
	keepspaces=true,
	numbers=left,
	numbersep=5pt,
	numberstyle=\scriptsize\color{darkred},
	showspaces=false,
	showstringspaces=false,
	showtabs=false,
	tabsize=2
}
\lstdefinestyle{genStyle}{
    basicstyle = {\ttfamily \small \color{darkgrey}},
    backgroundcolor = {\color{backcolor}},
	commentstyle=\color{codegreen},
    stringstyle = {\color{string-color}},
    keywordstyle = {\color{codeblue}},
	breakatwhitespace=false,
	breaklines=true,
	captionpos=b,
	keepspaces=true,
	numbers=left,
	numbersep=5pt,
	numberstyle=\scriptsize\color{darkred},
	showspaces=false,
	showstringspaces=false,
	showtabs=false,
	tabsize=2
}
\lstset{style=genStyle}
\lstdefinelanguage{CMake}{%
	morekeywords={if, else, endif, project, cmake_minimum_required, set, find_package, add_executable},%
	sensitive=false,%
	morecomment=[l]{\#},%
	morecomment=[s]{/*}{*/},%
	morestring=[b]",%
	otherkeywords={add_flows, setup_application},%
	keywordstyle = [2]{\color{codepurple}},
	morekeywords = [2]{REQUIRED, HINTS, VERSION, SYSTEM},
}
\lstdefinelanguage{Bash}{%
	morekeywords={if, else, fi, mkdir, cd, cmake, bash, git},%
	sensitive=false,%
	morecomment=[l]{\#},%
	morecomment=[s]{/*}{*/},%
	morestring=[b]",%
    keywordstyle = {\color{codeblue}},
    keywordstyle = [2]{\color{codepurple}},
    morekeywords = [2]{$, /, ..},
}

% Mathematica code
\usepackage{mmacells}
\mmaDefineMathReplacement[≤]{<=}{\leq}
\mmaDefineMathReplacement[≥]{>=}{\geq}
\mmaDefineMathReplacement[≠]{!=}{\neq}
\mmaDefineMathReplacement[→]{->}{\to}[2]
\mmaDefineMathReplacement[⧴]{:>}{:\hspace{-.2em}\to}[2]
\mmaDefineMathReplacement{∉}{\notin}
\mmaDefineMathReplacement{∞}{\infty}
\mmaDefineMathReplacement{𝕕}{\mathbbm{d}}
\mmaSet{
	leftmargin=4.5em,
	morefv={gobble=0},
	linklocaluri=mma/symbol/definition:#1,
	morecellgraphics={yoffset=1.9ex},
	labelsep=0.5em
}

% tikz stuff
\usepackage{tikz}
\usetikzlibrary{backgrounds}
\usetikzlibrary{decorations.pathreplacing}
\usetikzlibrary{shapes, arrows,calc}
\tikzstyle{roundbox} = [rectangle, draw, text centered, rounded corners,
minimum height=2em, minimum width=2em, draw=black!10, fill=blue!4]
\tikzstyle{process} = [rectangle, draw, minimum height=1em,
minimum width=3em, text centered, draw=black!10, fill=green!4]
\tikzstyle{integration} = [ellipse, draw, text centered, minimum height=1em,
minimum width=3em, draw=black!10, fill=red!4]
\tikzset{font={\fontsize{9pt}{11}\selectfont}}

% other
\usepackage{booktabs}
\usepackage{multirow}
\newcommand{\ra}[1]{\renewcommand{\arraystretch}{#1}}
\newcolumntype{C}{>{$}c<{$}}
\AtBeginDocument{
	\heavyrulewidth=.08em
	\lightrulewidth=.05em
	\cmidrulewidth=.03em
	\belowrulesep=.65ex
	\belowbottomsep=0pt
	\aboverulesep=.4ex
	\abovetopsep=0pt
	\cmidrulesep=\doublerulesep
	\cmidrulekern=.5em
	\defaultaddspace=.5em
}

%%%%%%%%%%%%% Macros %%%%%%%%%%%%%
\newcommand{\DiFfRG}{\texttt{DiFfRG}\xspace}
\newcommand{\cpp}[1]{\lstinline[language=C++,style=myStyle]|#1|}
\newcommand{\cmake}[1]{\lstinline[language=CMake]|#1|}
\newcommand{\mathem}[1]{\lstinline[language=Mathematica]|#1|}
\newcommand{\bash}[1]{\lstinline[language=Bash]|#1|}
\newcommand{\LEGO}{LEGO\textsuperscript{\textregistered}}

%%%%%%%%%%%%% Options %%%%%%%%%%%%%
\captionsetup{justification=centerlast}
\sisetup{range-units=single}

%%%%%%%%%% Table of Contents %%%%%%%%%%%
\makeatletter
%\def\l@subsubsection#1#2{}
\makeatother

%%%%%%%%%%%%% Graphic paths %%%%%%%%%%%%%
\graphicspath{{./figures/}}


%%%%%%%%%%%%% Title and hypersetup %%%%%%%%%%%%%
\newcommand{\gettitle}{DiFfRG 2.0}

\hypersetup{
	pdftitle={\gettitle},
	pdfauthor={Sattler},
	pdfkeywords={discontinuous galerkin} {finite element method} {functional renormalisation group} {effective potential} {phase transition} {numerical methods} {phase structure} {GPU computing} {parallel computing} {vertex expansion} {derivative expansion},
	bookmarksopen=true,
	bookmarksopenlevel=2,
	bookmarksnumbered=true,
	colorlinks,
	linkcolor={red!75!black},
	citecolor={blue!75!black},
	urlcolor={blue!75!black}
}

\journal{Computer Physics Communications}

%%%%%%%%%%%%% Document %%%%%%%%%%%%%
\begin{document}

\renewcommand{\thefootnote}{\fnsymbol{footnote}}
\begin{frontmatter}

	\title{\gettitle}
	
	\author[a]{Keiwan Jamaly
		\footnote{\url{yourmail}}}

	\author[c]{Franz R. Sattler 
		\footnote{\url{fsattler@physik.uni-bielefeld.de}}}


	\address[a]{Institut f\"ur Theoretische
		Physik, Universit\"at Heidelberg, Philosophenweg 16, 69120 Heidelberg,
		Germany}
	\address[b]{ExtreMe Matter Institute EMMI, GSI, Planckstr. 1, 64291 Darmstadt, Germany}
	\address[c]{Institut f\"ur Physik, Universit\"at Bielefeld...}

	\begin{abstract}

		We update \texttt{DiFfRG} (\texttt{Di}scretisation \texttt{F}ramework for \texttt{f}unctional \texttt{R}enormalisation \texttt{G}roup flows).

	\end{abstract}

\end{frontmatter}
\renewcommand{\thefootnote}{\arabic{footnote}}

%%%%%%%%%%%%%%%%%%%%
\vspace{0.25cm}
{\bf PROGRAM SUMMARY}\\\vspace{-0.2cm}

\begin{small}
	\noindent
	{\em Program Title:}  DiFfRG \\
	{\em Licensing provisions:} GPLv3                                   \\
	{\em Programming language:} C++, Mathematica, Python                                  \\
	{\em Computer:}  Any Linux/Mac machine                           \\
	{\em Nature of problem:}  Evaluate large systems of fRG flows with full effective potentials and/or vertex expansions\\
	{\em Solution method:} Combine finite element methods and efficient time-steppers with fast numerical integration of flow equations and momentum grids. \\
	{\em Unusual features:} The code-generation capabilities of the \texttt{Mathematica} subpackage allow for very fast development of extremly large systems of flow equations. Field-space discretisations can be easily defined by the user with a range of different finite element methods.\\
\end{small}

%\hrule
\clearpage
\tableofcontents


%%%%%%%%%%%%%%%%%%%%
\section{Introduction}
\label{sec:introduction}

Mention the previous work...

Things we want to highlight...

Overview...


%%%%%%%%%%%%%%%%%%%%
\section{Example applications}
\label{sec:examples}
%
\noindent
%
All examples are  present in the folder \texttt{Examples/} and can be built using the scripts \texttt{build.sh}, present in every single subfolder, corresponding to the below examples.


%%%%%%%%%%%%%%%%%%%%
\subsection{Example 01}
\label{sec:example01}

We want some examples, benchmarks, etc


%%%%%%%%%%%%%%%%%%%%
\section{Summary and outlook}
\label{sec:outlook}

The \DiFfRG framework is still evolving.


%%%%%%%%%%%%%%%%%%%%
\section*{Acknowledgments}

We thank Andreas Geissel, Friederike Ihssen and Nicolas Wink for discussion and collaboration on related projects.
This work is done within the fQCD collaboration \cite{fQCD} and we thank its members for discussions and collaborations  on related projects.
This work is funded by the Deutsche Forschungsgemeinschaft (DFG, German Research Foundation) under Germany’s Excellence Strategy EXC 2181/1 - 390900948 (the Heidelberg STRUCTURES Excellence Cluster) and the Collaborative Research Centre SFB 1225 (ISOQUANT).
FRS acknowledges funding by the GSI Helmholtzzentrum f\"ur Schwerionenforschung.


%%%%%%%%%%%%%%%%%%%%


\appendix
%% now insert the new code block
\gdef\thesection{\Alph{section}}
\makeatletter
\renewcommand\@seccntformat[1]{\appendixname\ \csname the#1\endcsname.\hspace{0.5em}}
\makeatother
\begingroup
\allowdisplaybreaks

%%%%%%%%%%%%%%%%%%%%
\section{First}

\endgroup
%%%%%%%%%%%%%%%%%%%%

\bibliographystyle{elsarticle-num}
\bibliography{ref-lib}

\end{document}